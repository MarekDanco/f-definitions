\documentclass{llncs}
\usepackage[textwidth=3cm,tickmarkheight=3pt]{todonotes}
\usepackage[utf8]{inputenc}
\usepackage{amssymb, amsmath}
\usepackage{numprint}
\usepackage{algorithm}
\usepackage[noend]{algpseudocode}
\usepackage{color}
\definecolor{citeblue}{rgb}{0.1,0,.4}
\usepackage[pdftex%
,colorlinks=true%
,bookmarks=true%
,linkcolor=citeblue%
,citecolor=citeblue%
,urlcolor=blue%
,plainpages=false]{hyperref}
\hypersetup{colorlinks,citecolor=blue,linkcolor=blue,urlcolor=blue}
\AtBeginDocument{\hypersetup{pdfborder={0 0 1}}}
\usepackage[nameinlink,capitalize,noabbrev]{cleveref} % must come after hyperref
\usepackage{commath}
\usepackage[most]{tcolorbox}
\usepackage{enumitem}
\usepackage{orcidlink}

\newcommand{\sygus}{SyGuS\xspace}
\newcommand{\qedhere}{$\blacksquare$}
\newcommand{\numvar}[1]{|#1|}


\renewcommand{\algorithmicrequire}{\textbf{Input:}}
\renewcommand{\algorithmicensure}{\textbf{Output:}}

%\newtheorem{definition}{Definition}
%\newtheorem{property}{Property}
%\newtheorem{example}{Example}

%\newtheorem{theorem}{Theorem}
%\newtheorem{lemma}[theorem]{Lemma}
%\newtheorem{corollary}[theorem]{Corollary}
%\theoremstyle{remark}
%\newtheorem{claim}{Claim}
\usepackage{etoolbox}
%\AtEndEnvironment{proof}{\setcounter{claim}{0}}

\npthousandsep{,}
\newcommand{\gmodel}{\mathcal{M}}
\newcommand{\smt}{\textsc{smt}}
\newcommand{\args}{\text{args}}
\usepackage{xspace}
\newcommand{\req}{\emph{ReqPivot}\xspace}
\newcommand{\subst}[3]{#1[#2\leftarrow #3]}
\newcommand{\repr}[1]{\hat{#1}}

\newcommand{\todoSR}[1]{\todo[color=blue!40]{#1}}
\newcommand{\mj}[1]{\todo[color=green!30]{#1 -mik}}
\newcommand{\marek}[1]{\todo[color=red!10]{#1 -MD}}
\newcommand{\todoSN}[1]{\todo[color=orange!50]{#1 -SN}}


\begin{document}



The following derives the formulas used by the algorithm directly from the definition of satisfiability certificate and the definition of $\preceq$ and $X_z$ used by the algorithm. One might even start by only assuming $\preceq$ to be defined by a LIA expression and $X_z$ to be defined using finitely many cases, where the condition of each case is again defined using a LIA expression.

The definitions of $\preceq$ and $X_z$ used by the algorithm are:
\begin{itemize}
  
\item $x\preceq x'$ iff
\[ |x|<|x'| \vee (|x|=|x'|\wedge sgn(x)>sgn(x')) \]
\item 
    \[
    X_z = 
    \begin{cases}
    \emptyset & \text{if } z \in B, \\[4pt]
    \bigl\{ f(I(\subst{t}{x}{z})) \mid f(t)\in\mathcal{S} \bigr\} & \text{if } z > b_{\max}, \\[4pt]
    \bigl\{ f(I(\subst{t}{x}{z})) \mid f(t)\in\mathcal{S}' \bigr\} & \text{if } z < b_{\min}.
    \end{cases}
  \]
\end{itemize}


condition used by definition of satisfiability certificate
  

    \[X_z\cap (\mathit{def}(I)\cup\bigcup_{z'\prec z} \Gamma_{I,z'}(\phi))=\emptyset,\]

split into clash condition and propagation condition:    
    \[X_z\cap \mathit{def}(I)=\emptyset, X_z\cap\bigcup_{z'\prec z} \Gamma_{I,z'}(\phi)=\emptyset,\]   

    clash condition:
    \[
      \forall z\;.\; z\not\in B \Rightarrow \bigwedge_{f(t)\in \mathcal{S}}\bigwedge_{a\in args(F,f) } a\neq t[x\leftarrow z] 
  \]

propagation condition:  
    
\[
  \forall z', z\;.\; z'\prec z \wedge z\not\in B \Rightarrow \bigwedge_{f(t')\in \mathcal{T}, f(t)\in S} t'[x\leftarrow z'] \neq t[x\leftarrow z]
  \]

        \[\forall x \forall \bar{u} \;.\; congu\Rightarrow \exists \bar{v}.\ congv\wedge Q'(\mathcal{S}) \]
        where $\bar{u}$ corresponds to the variables $u_{f(t)}$ with $f(t)\in\mathcal{T}\setminus\mathcal{S}$ and $\bar{v}$ to the variables $v_{f(t)}$ with $f(t)\in \mathcal{S}$, and $Q'$ is obtained from $Q$ by simultaneously replacing
        \begin{itemize}
            \item every $f(t)\in\mathcal{T}\setminus\mathcal{S}$ by $u_{f(t)}$, and
            \item every $f(t)\in\mathcal{S}$ by $v_{f(t)}$.
            \end{itemize}

congu
\[\bigwedge_{ f(t)\in \mathcal{T}\setminus\mathcal{S},f(t')\in \mathcal{T}\setminus\mathcal{S}} t=t'\Rightarrow u_{f(t)}=u_{f(t')}\]

congv
\[  \bigwedge_{ f(t)\in \mathcal{T},f(t')\in \mathcal{T}} t=t'\Rightarrow v_{f(t)}=v_{f(t')}\]



%\bibliographystyle{abbrv}
%\bibliography{refs}



\appendix




\end{document}

%%% Local Variables:
%%% mode: latex
%%% TeX-master: t
%%% End:
