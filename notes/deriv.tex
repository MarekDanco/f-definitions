\documentclass{llncs}
\usepackage[textwidth=3cm,tickmarkheight=3pt]{todonotes}
\usepackage[utf8]{inputenc}
\usepackage{amssymb, amsmath}
\usepackage{numprint}
\usepackage{algorithm}
\usepackage[noend]{algpseudocode}
\usepackage{color}
\definecolor{citeblue}{rgb}{0.1,0,.4}
\usepackage[pdftex%
,colorlinks=true%
,bookmarks=true%
,linkcolor=citeblue%
,citecolor=citeblue%
,urlcolor=blue%
,plainpages=false]{hyperref}
\hypersetup{colorlinks,citecolor=blue,linkcolor=blue,urlcolor=blue}
\AtBeginDocument{\hypersetup{pdfborder={0 0 1}}}
\usepackage[nameinlink,capitalize,noabbrev]{cleveref} % must come after hyperref
\usepackage{commath}
\usepackage[most]{tcolorbox}
\usepackage{enumitem}
\usepackage{orcidlink}

\newcommand{\sygus}{SyGuS\xspace}
\newcommand{\qedhere}{$\blacksquare$}
\newcommand{\numvar}[1]{|#1|}


\renewcommand{\algorithmicrequire}{\textbf{Input:}}
\renewcommand{\algorithmicensure}{\textbf{Output:}}

%\newtheorem{definition}{Definition}
%\newtheorem{property}{Property}
%\newtheorem{example}{Example}

%\newtheorem{theorem}{Theorem}
%\newtheorem{lemma}[theorem]{Lemma}
%\newtheorem{corollary}[theorem]{Corollary}
%\theoremstyle{remark}
%\newtheorem{claim}{Claim}
\usepackage{etoolbox}
%\AtEndEnvironment{proof}{\setcounter{claim}{0}}

\npthousandsep{,}
\newcommand{\gmodel}{\mathcal{M}}
\newcommand{\smt}{\textsc{smt}}
\newcommand{\args}{\text{args}}
\usepackage{xspace}
\newcommand{\req}{\emph{ReqPivot}\xspace}
\newcommand{\subst}[3]{#1[#2\leftarrow #3]}
\newcommand{\repr}[1]{\hat{#1}}

\newcommand{\todoSR}[1]{\todo[color=blue!40]{#1}}
\newcommand{\mj}[1]{\todo[color=green!30]{#1 -mik}}
\newcommand{\marek}[1]{\todo[color=red!10]{#1 -MD}}
\newcommand{\todoSN}[1]{\todo[color=orange!50]{#1 -SN}}


\begin{document}



The following derives the formulas used by the algorithm directly from the definition of satisfiability certificate and the definition of $\preceq$ and $X_z$ used by the algorithm. Assume a predicate $\preceq$ defined using LIA. In our current approach $x\preceq x'$ iff
\[ |x|<|x'| \vee (|x|=|x'|\wedge sgn(x)>sgn(x')). \]
The conditions used by definition of satisfiability certificate (split into clash condition and propagation condition) are:
\[X_z\cap \mathit{def}(I)=\emptyset, X_z\cap\bigcup_{z'\prec z} \Gamma_{I,z'}(\phi)=\emptyset.\]
Now assume a partitioning of the quantification space using predicates $cond_1,\dots, cond_n$ such that $\models\bigvee_{i\in \{1,\dots,n\}} cond_i$ and for all $i,j\in \{1,\dots, n\}$ with $i\neq j$, $cond_i\wedge cond_j$ is unsatisfiable. In our current approach we use an interval $B=[b_{\min}, b_{\max}]$ and $cond_1\equiv z\in B$, $cond_2\equiv z>b_{\max}$, $cond_3\equiv z<b_{\min}$.


Assume $\mathcal{S}_1,\dots,\mathcal{S}_n\subseteq \mathcal{T}$ such that \[\bigwedge_{i\in\{1,\dots, n\}} cond_i\Rightarrow X_z=\{ f(I(\subst{t}{x}{z}))\mid f(t)\in S_i\}.\] In our current approach $\mathcal{S}_1=\emptyset$ and $\mathcal{S}_2$ and $\mathcal{S}_3$ are named $\mathcal{S}$ and $\mathcal{S}'$, and we do not used fixed sets, but solve for them.

Then the clash condition becomes   
    \[
      \forall z\;.\; \bigwedge_{i\in \{1,\dots, n\}} cond_i(z)\Rightarrow \bigwedge_{f(t)\in \mathcal{S}_i}\bigwedge_{a\in args(F,f) } a\neq t[x\leftarrow z] 
  \]
and the propagation condition
\[
  \forall z', z\;.\; z'\prec z \Rightarrow \bigwedge_{i\in \{1,\dots, n\}} cond_i(z) \Rightarrow\bigwedge_{f(t')\in \mathcal{T}, f(t)\in S_i} t'[x\leftarrow z'] \neq t[x\leftarrow z]
\]
Based on this, we can use a more precise ReqPivot that takes into account the information whether two arguments are the same (and hence refer to the same cell) or not:
\[\forall z\bigwedge_{i\in \{1,\dots, n\}}  cond_i(z)\Rightarrow\forall \bar{u} \;.\; congu_i\Rightarrow \exists \bar{v}.\ congv_i\wedge Q'(S_i) \]
where $congu_i$ is
\[\bigwedge_{ f(t)\in \mathcal{T}\setminus\mathcal{S}_i,f(t')\in \mathcal{T}\setminus\mathcal{S}_i} t=t'\Rightarrow u_{f(t)}=u_{f(t')},\]
$congv_i$ is
\[  \bigwedge_{ f(t)\in \mathcal{S}_i,f(t')\in \mathcal{S}_i} t=t'\Rightarrow v_{f(t)}=v_{f(t')}\]
and where $\bar{u}$ corresponds to the variables $u_{f(t)}$ with $f(t)\in\mathcal{T}\setminus\mathcal{S}_i$ and $\bar{v}$ to the variables $v_{f(t)}$ with $f(t)\in \mathcal{S}_i$, and $Q'(S_i)$ is obtained from $Q$ by simultaneously replacing
        \begin{itemize}
            \item every $f(t)\in\mathcal{T}\setminus\mathcal{S}_i$ by $u_{f(t)}$, and
            \item every $f(t)\in\mathcal{S}_i$ by $v_{f(t)}$.
            \end{itemize}

%\bibliographystyle{abbrv}
%\bibliography{refs}



\appendix




\end{document}

%%% Local Variables:
%%% mode: latex
%%% TeX-master: t
%%% End:
