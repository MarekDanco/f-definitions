\documentclass[a4paper,11pt]{article}

\newtheorem{definition}{Definition}
\newtheorem{example}{Example}

\begin{document}

\section{Basic Setup}

Goal: Proving satisfiability of UFLIA formulas (or UFNIA) formulas (which may prove unsat on the way).

In the following, a formula will always refer to a UFNIA formula whose only quantifiers are universal quantifiers over integer variables.

\begin{definition}
    An \emph{entity} of a formula is either a numerical variable or a pair consisting of an  uninterpreted function symbol and an integer (which models an index of the uninterpreted function symbol).
\end{definition}

Idea: View quantifiers as something that expresses infinitely many constraints on entities. The problem is that a single entity may occur in several such constraints and we must avoid assigning different values to the very same entity. To prove satisfiability of a formula, order these constraints and use a symbolic representation of the satisfying assignment that assigns values to entities in such a way that following the order of constraint avoids assigning different values to the very same entity.

\begin{definition}
    Given a formula $\phi$, a \emph{satisfiability certificate} is
    \begin{itemize}
        \item an order\footnote{For now, I assume this order to be total.} on all instantiations of quantified sub-formulas of $\phi$, viewing the quantifier-free part as ONE quantified formula with an empty tuple of quantified variables, hence having one instantiation, and
        \item for each instantiation of a quantified formula
              \begin{itemize}
                  \item a subset of its entities marked as existential such that the existential entities do not occur in any formula instantiation occurring before according to the given order, and
                  \item a certificate showing that for all values of the non-existential entities there exist values for all other entities satisfying this instantiation, which we will call \emph{satisfiability propagator}.
              \end{itemize}
    \end{itemize}
\end{definition}

The satisfiability propagators typical have the form entity=term, where the left-hand sides are the existential  variables\footnote{At some point, we might allow satisfiability propagators of a different form, for example, to represent real-algebraic number solutions in the case of the reals}.


\begin{example}
    \[ \forall x \:.\: g(x+2)=g(x)+1 \]
    \begin{itemize}
        \item order: $0,1,\dots, -1, -2, \dots$
        \item corresponding existential entities: $g(2), g(3), \dots, g(-1), g(-2), \dots$   (note that g(0) and g(1) are not existential anywhere, and hence can be chosen arbitrarily)
        \item satisfiability propagators for the two blocks in the order: $g(x)=g(x-2)+1$, $g(x)= g(x+2)-1$
    \end{itemize}

\end{example}

\begin{example}
    \[f(3, a) \geq 4 + g(b) \wedge (\forall x, y \:.\: f(x, y) <x + g(y)) \wedge (\forall x \:.\: g(x) = g(x+2))\]
    \begin{itemize}
        \item order on constraints, block-wise:
              \begin{enumerate}
                  \item     $f(3, a) \geq 4 + g(b)$,
                  \item $f(x,y) <x+g(y)$ for $(x,y)= (3,0)$
                  \item $g(x) = g(x + 2)$ for $x= 0, 1, \dots$
                  \item $g(x) = g(x + 2)$, for $x= -1, -2, \dots$,
                  \item  any order on the rest of $\forall x,y \:.\: f(x,y)<x+g(y)$ (i.e., $(x,y) \not=(3,0)$)
              \end{enumerate}

        \item corresponding existential entities:
              \begin{enumerate}
                  \item  $\{a, b, f(3,0), g(1)\}$
                  \item $\{ g(0)\}$
                  \item $\{ g(2) \}, \{ g(3) \}, \dots$
                  \item $\{ g(-1) \}, \{ g(-2) \}, \dots$
              \end{enumerate}

        \item satisfiability propagators:
              \begin{enumerate}
                  \item $a=0, b=1, f(3,0)=4, g(1)=0$
                  \item $g(0)=2$
                  \item $g(x)= g(x-2)$
                  \item $g(x)=g(x+2)$
                  \item $f(x,y)=x+g(y)-1$
              \end{enumerate}

    \end{itemize}



\end{example}


\section{Constructing Certificates}

Very rough idea:
\begin{enumerate}
    \item solve quantifier-free part
    \item check solution against instantiations of quantified part with all entities already assigned, if not satisfied, learn a clause and backtrack to solving
    \item order all other instantiations and construct the necessary certificates

\end{enumerate}


% *** example:

% attempt 1:
% - solve part without quantifiers: a=0, b=0, g(b)=g(0)=0, f(3,a)=f(3,0)=4, then f(3,0)>= 4+g(0)
% - this violates  (\forall 𝑥, 𝑦. 𝑓 (𝑥, 𝑦) < 𝑥 + 𝑔(𝑦)) since for x=3, y=0 this is f(3,0)<3+g(0), since not f(3,0)=4<3=3+g(0)
% - backtrack, learn f(3,0)<3+g(0), which excludes attempt 1 (clause learning might learn something more general, e.g., including a and b)
% attempt 2:
% - solve (𝑓 (3, 𝑎) ≥ 4 + 𝑔(𝑏)), f(3,0)<3+g(0) resulting in  a=0, b=1, g(b)=g(1)=0, f(3,a)=f(3,0)=4, g(a)=g(0)=2
% - now try to satisfy \forall 𝑥. 𝑔(𝑥) = 𝑔(𝑥 + 2):
%     for x in { 0, 1 }, this is satisfied by the assignment already constructed
%     for all x>=2, set g(x)=g(x-2)
%     for all x<0, set g(x)=g(x+2)
% - now try to satisfy (\forall 𝑥, 𝑦. 𝑓 (𝑥, 𝑦) < 𝑥 + 𝑔(𝑦)):
%     for x=3,y=0 this is satisfied by the assignment already constructed: f(3,0)=4<5=3+2=3+g(0) that is 4<3+2
%     for all other x,y: set f(x,y)=x+g(y)-1



\end{document}


%%% Local Variables:
%%% mode: latex
%%% TeX-master: t
%%% End:
