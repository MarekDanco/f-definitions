%!TeX program = latexmk
\documentclass{easychair}
\usepackage{todonotes}
\usepackage{doc}
\usepackage[utf8]{inputenc}
\usepackage{amssymb}
\usepackage{numprint}
\npthousandsep{,}

\newcommand{\gmodel}{\mathcal{M}}

\title{
    Iterative Function Definitions
}
\author{Stefan Ratschan \and Marek Dančo \and Mikoláš Janota}
\authorrunning{Dančo et al.}
\titlerunning{Iterative Function Definitions}
\institute{
}

\begin{document}

\maketitle

\section*{Motivation}
Consider the following UFLIA formula:
\[
    \varphi =
    \underbrace{(f(3, a) \geq 4 + g(b))}_{\varphi_G}\,\land \,
    \underbrace{(\forall x, y. \,f(x, y) < x + g(y))}_{\varphi_1}\,\land \,
    \underbrace{(\forall x. \,g(x) = g(x + 2))}_{\varphi_2}
\]
Note that currently, the SMT solver Z3 cannot prove $\varphi$ is satisfiable.
However, we show there is an algorithmic way of proving the satisfiability
of this formula and try to identify UFLIA fragments in which this algorithm is applicable.

Let $\gmodel$ be the following partial model of the ground part $\varphi_G$:
\begin{gather*}
    a^\gmodel = 1, \\
    b^\gmodel= 0, \\
    f^\gmodel= \{(3, 1) \mapsto 5\}, \\
    g^\gmodel= \{0 \mapsto 1\}.
\end{gather*}
The definitions of $f^\gmodel, g^\gmodel$ on the rest of the domain are not interesting at this point.
Notice that because of the structure of $\varphi_2$ and the fact that $g^{\gmodel}(0) = 0$,
we can now define $g^{\gmodel}$ on all even integers.
Specifically, we can first order the arguments of $g$ in $\varphi_2$ with $x \prec x+2$, and define
$g(2), g(4), g(6), \dots$ by unifying $x$ with $0, 2, 4, \dots$ and then set $x+2 \prec x$ and define
$g(-2), g(-4), g(-6), \dots$ by unifying $x+2$ with $0, -2, -4, \dots$ That is, we can define
$g(A_2)$ using $g(A_1)$ if we have $A_1 \prec A_2$.
This way we can expand $\gmodel$ to also be a partial model of $\varphi_2$.

Since $f^{\gmodel}(3,1) = 5$ and it must hold that $f(3,1) < 3 + g(1)$, we can set $g^{\gmodel}(1) = 3$.
Just like above, we can now define $g^{\gmodel}$ on all odd integers and thus obtain a total
interpretation of $g$.
Now, for any pair $(i, j) \not= (3, 1)$ of integers, we can define $f^{\gmodel}(i, j)$ to be
$i - 1$ or $i + 2$ for even or odd $j$ respectively. In the end, we can expand $\gmodel$
to a model of $\varphi$.

Observe it must be that $a \not= b$, otherwise
$f(3, a) \geq 4 \,+ \,g(b) \,\land \,f(3, a) < 3 \,+ \,g(a)$ is unsatisfiable.
Consequently, $\gmodel$ must be such that $a^{\gmodel} \not= b^{\gmodel}$.
We can check the compatibility of $\gmodel$ with the rest of the formula
after each step of our procedure and when a conflict is found,
we can learn the new ground literals from the \textsc{unsat} core and
restart the procedure with the updated ground part.

\section*{Algorithm}

\end{document}
